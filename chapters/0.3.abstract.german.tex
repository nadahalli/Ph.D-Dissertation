\chapter*{Zusammenfassung}
Im Januar 2009 schuf Satoshi Nakamoto den Bitcoin \cite{bitcoin_whitepaper}. Als neue Form von Geld, das nicht von Nationalstaaten ausgegeben oder kontrolliert wird, muss Bitcoin in vielerlei Hinsicht Kompromisse eingehen: Überleben, Dezentralisierung, Zensurresistenz, Skalierbarkeit, Datenschutz und mehr. Auf jeder dieser Ebenen haben Forscher im Laufe der Jahre viele Fragen zu Bitcoin gestellt, und es scheint, dass Bitcoin in der Theorie nicht funktionieren sollte. Entgegen aller Erwartungen scheint es jedoch in der Praxis zu funktionieren - alle 10 Minuten wird ein Block erzeugt.

Wir glauben, dass die Beantwortung eines kleinen Teils dieser Forschungsfragen den Bitcoin-Anhängern, die glauben, dass Bitcoin die Welt zum Besseren verändern wird, etwas Erleichterung verschaffen wird. Vielleicht stoßen wir sogar ehrliche Skeptiker dazu an, tiefergehende Fragen zu stellen.

Bitcoin behauptet, ein zensurresistentes Geldsystem zu bieten. In dieser Arbeit zeigen wir, dass eine bestimmte Klasse von Transaktionen anfällig für Zensur ist, aber nicht tatsächlich zensiert wird. Unsere Arbeit gibt eine Antwort auf die Frage, warum das so ist, und weist auf eine intrinsische Beziehung zwischen schwachen Minern und der (Un-)Fähigkeit von Alice hin, Anreize für die Zensur von Bobs Transaktion zu schaffen.

Nutzer können ihre Privatsphäre in Bitcoin erhöhen, indem sie ihre Münzen untereinander tauschen. Coin-Swapping-Protokolle neigen dazu, Coins zu sperren, was zu Opportunitätskosten führt. In dieser Arbeit schlagen wir einen kummerfreien atomaren Tausch vor, der diese Opportunitätskosten minimiert. 

Das Lightning Network skaliert Bitcoin als Zahlungssystem, indem es ein Netzwerk von Kanälen hat. In dieser Arbeit schlagen wir eine neue Kanalstruktur vor, die das Netzwerk robuster macht. Zahlungskanäle hängen davon ab, dass die Nutzer online sind, um den Kanalvertrag auf der Blockchain durchzusetzen, falls jemand betrügt. Offline-Nutzer setzen eine dritte Partei, einen sogenannten Wachturm, ein, um ihre Kanäle zu überwachen und Betrug zu verhindern. Unsere neue Lightning-Channel-Struktur ermöglicht effiziente Wachtürme, indem sie deren Speicherkosten drastisch reduziert. 

Bitcoin ist ein geschlossenes, selbstverwaltendes System, bei dem der externe Dateninput minimiert ist. Zustandsabhängige Blockchains wie Ethereum haben intelligente Verträge, die sich auf externe Daten wie Marktpreise von Vermögenswerten stützen. Diese sind auf triviale Weise Angriffen durch Orakel ausgesetzt, die die Datenquelle kontrollieren. Diese Angriffe können durch die Verwendung einer intrinsischen Quelle externer Daten, wie z. B. dem Preis eines automatisierten Marktmachers für einen Vermögenswert, entschärft werden. In dieser Arbeit zeigen wir, dass solche intrinsischen Datenquellen billig manipuliert werden können, was zu schlechten Ergebnissen für ihre Nutzer führt. Diese Art von Angriffen unterstreicht die konservative Kultur von Bitcoin mit minimalen, aber sicheren intelligenten Verträgen - im Gegensatz zu reichhaltigen, aber angreifbaren intelligenten Verträgen auf anderen Plattformen. Indem Bitcoin seine Smart Contracts frei von globalen Zuständen und externen Datenquellen hält, optimiert es sein langfristiges Überleben.