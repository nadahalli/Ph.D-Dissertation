% The official guideline for the length of the abstract is 1.5 A4 pages.

\chapter*{Abstract}

In January 2009, Satoshi Nakamoto created Bitcoin \cite{bitcoin_whitepaper}. As a new form of money not issued or controlled by nation states, Bitcoin makes trade-offs along many axes: survivability, decentralization, censorship-resistance, scalability, privacy, and more. Along each of these axes, over the years, researchers have asked many questions of Bitcoin and it seems that Bitcoin should not work in theory. Against all odds though, it seems to be working in practice - generating a block every 10 minutes.

We believe that answering a small fraction of these research questions will give some relief to believers of Bitcoin's potential to change the world for the better. It might even nudge honest skeptics towards asking deeper questions.

Bitcoin claims to offer a censorship-resistant monetary system. In this thesis, we show that a certain class of transactions are vulnerable to censorship, but are not actually getting censored. Our work points to an intrinsic relationship between weak miners and Alice's (in)ability to incentivize censorship of Bob's transaction.

Users can increase their privacy in Bitcoin by swapping their coins with each other. Coin swapping protocols tend to lock up coins, leading to opportunity cost. In this thesis, we propose grief-free atomic swaps, which minimizes this opportunity cost. 

The Lightning Network scales Bitcoin as a payment system by having a network of channels. In this thesis, we propose a new channel structure that makes the network more robust. Payment channels depend on users being online to enforce the channel contract on the blockchain in case someone cheats. Offline users employ a third party, called a watchtower, to monitor their channels and prevent cheating. Our new lightning channel structure enables efficient watchtowers by dramatically reducing their storage costs. 

Bitcoin is a closed self-governing system where extrinsic data input is minimized. Stateful blockchains like Ethereum have smart contracts that rely on extrinsic data like market price of assets. These are trivially subjected to attacks by oracles who control the data-source. Some smart contracts have tried to mitigate this by using an intrinsic source of external data, like an automated market maker's market price of an asset. In this thesis, we show that such intrinsic data-sources can be manipulated cheaply leading to bad outcomes for their users. These kind of attacks highlight Bitcoin's conservative culture of minimal, but safer smart contracts - as opposed to rich, but vulnerable smart contracts in other platforms.