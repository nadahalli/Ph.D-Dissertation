\chapter{Timelocked Bribing}  % Motivation
\markboth{Timelocked Bribing}{}
\label{chapter:bribing}

\section{Introduction}
In traditional finance, transactions are finalized by institutions who give authoritative statements that certain transactions are final, and users can take these institutional guarantees as evidence in a court of law if there are disputes. Similarly, if someone is not allowed to transact on platforms, that is, they are censored, they can ask redressal from a court of law. If the censorship is enacted by the government in question, the user has little recourse.

As an alternative medium of exchange, Bitcoin removes trusted financial intermediaries and replaces them with a dynamic set of miners. These miners validate transactions and are paid by the system in the form of block rewards and also by transaction participants in the form of fees. The entire set of miners, collectively, have no incentive to censor any particular transaction. Even if governments wanted to censor some transactions, there is no easy way to get this message across to all miners. Any miner including such a transaction in their block is enough to thwart the censorship. More than 50\% of miners have to abandon that block to effectively censor the transaction. In other words, Bitcoin achieves censorship resistance if more than 50\% miners act rationally. 

In this chapter, we look at whether there are some Bitcoin transactions that rational miners might be incentivized to censor. Rational miners will always choose higher-fee transactions than lower-fee ones, and this behavior will get reinforced over time as block rewards decrease to zero \cite{bonneau2015sok}. This setup has often raised (\cite{feather_forking} \cite{whale_transactions} \cite{smart_contracts_for_bribing}) the possibility of miners being bribed by transaction participants to favor one participant over the other. Typical bribing attacks envision the paying party (Alice) cheating the paid party (Bob) by Alice double-spending the same value in a separate transaction paying back to Alice. Miners are bribed by Alice to include the double-spending transaction in the blockchain by forking it and orphaning the block with the first transaction, thereby cheating Bob of the payment from the first transaction. These bribery attacks, however, operate at a block level because, to be cheated, Bob needs to be convinced that the first transaction is buried in the blockchain by $k$ blocks (in Bitcoin, $k=6$). Before this happens, Bob should ideally not honor the first transaction, but monitor the public Bitcoin blockchain. If a transaction where Alice double-spends the same bitcoins back to herself is seen, and Bob's transaction is abandoned in an orphaned block, Bob should not honor Alice's first transaction by not giving Alice the goods and services that were promised.

As we saw in the background section in Chapter \ref{chap:background}, HTLC's are a more sophisticated type of transaction where Bob \emph{does} want Alice to pay the transaction value back to herself, but only after some time has elapsed. During this time, Bob reserves the option of getting paid himself from the same payment source. HTLC's are the building blocks for financial contracts like escrows, payment channels, atomic swaps, etc. The required time delay is implemented using a blockchain artefact called \emph{timelocks}. A rudimentary version of timelocks (nLocktime) was in the first Bitcoin implementation by Satoshi Nakamoto in 2009 \cite{satoshi_bitcoin_release_0_1_0}. More sophisticated timelocks that lock transactions, specific bitcoins, or specific script execution paths were added later \cite{BIP68} \cite{BIP65} \cite{BIP112}. Bitcoin script allows for timelocks to be combined with hashlocks in an OR condition to create Hash Timelocked Transactions (HTLC). As we will see later, HTLC's open the possibility of transaction level bribing of miners where miners do not have to orphan mined blocks, but just have to ignore a \emph{currently valid} transaction and wait for the timelocked bribe to become valid. Additionally, in this attack, the bribe is endogenous to the transactions and does not have to be implemented externally through public bulletin boards or other third party smart contracts. Bribery attacks that operate at a transaction level are far more insidious compared to block orphaning bribery attacks. Block orphaning attacks undermine the native cryptocurrency's trust with the larger community and could be detrimental to the briber's financial position in general. Transaction level bribery, on the other hand, targets specific contracts on the blockchain and could go unnoticed as the larger cryptocurrency system hums along. This sort of an attack, where a miner has visibility into the pool of transactions that are waiting for confirmation (mempool) and can include or not include a transaction in their mined block is discussed in a more general setting in \cite{daian2020flash} under the umbrella term ``Miner Extractable Value''.

\subsection{Bribing Attack}

The attack can begin after the \htlctxn{} is confirmed and the buyer already has the goods/services for which the buyer committed the funds for. If the buyer acts in good faith and does nothing, there is no attack. If the buyer acts in bad faith, the buyer will try to censor \sellertxn{} from being included in any future block. The buyer broadcasts the \refundtxn{} (which sends the funds back to the buyer) and chains it with a \bribetxn{}, which sends the funds from the buyer to any miner who mines it by leaving the output field empty. Note that in the \bribetxn{}, the buyer can send an $\epsilon$ amount to themselves. This makes the bribe not just a griefing attack (where the attacker does not profit), but marginally profitable. Also note that \sellertxn{} and the pair [\refundtxn{}, \bribetxn{}] spend the same UTXO and are inherently incompatible. If one of them is confirmed on the blockchain, the other becomes invalid. In the rest of this paper, we will use \bribetxn{} and the pair [\refundtxn{}, \bribetxn{}] interchangeably. Pseudo-code for these transactions are in Figure \ref{fig:bribe_txn}.

\begin{figure}[!htb]
    \centering
    \caption{HTLC followed by Bribe}
    \label{fig:bribe_txn}
\begin{align*}
    \htlctxn{} &= [(P|\sigma_b)] \mapsto [(P|(\sigma_b \land \Delta) \lor (\sigma_s \land H_{s}))] \\
    \sellertxn{} &= [(P|(\sigma_s \land s))] \mapsto [(P|\sigma_s)] \\
    \refundtxn{} &= [(P|(\sigma_b\land \Delta))] \mapsto [(P|\sigma_b)] \\
    \bribetxn{} &= [(P|\sigma_b)] \mapsto [(\epsilon|\sigma_b)]
\end{align*}
\end{figure}

Bitcoin's consensus rules govern what transactions can be included in a block by miners, but does not say anything about what transactions miners can or cannot ignore. It gives the benefit of the doubt to miners, allowing the possibility that miners have not seen a specific transaction because of network delays/failures. Miners could be (or not be) interested in a transaction because its fees are high (or low). In our attack scenario, miners see \sellertxn{} and \bribetxn{} at the same time. But as per the consensus rules, miners cannot include \bribetxn{} immediately because it is timelocked. But crucially, there is no obligation to include the \sellertxn{} immediately either. As blocks go by, \bribetxn{} becomes valid and can be included in the blockchain and  \sellertxn{} is censored, with the sale proceeds going to the miners and the buyer, but not to the seller. The seller could increase their fees to compete with the timelocked bribe, but that would come out of their own pocket, as they have already handed out the goods and services to the buyer.

In the following sections, we show how the two main applications of HTLC's: Lightning Payment Channels and Atomic Swaps, are both vulnerable to this bribing attack.

\subsection{Payment Channels}
As we saw in the background section in Chapter \ref{chap:background}, payment channels \cite{decker_wattenhofer}, \cite{poon_dryja} are a promising solution to the scalability problem in Bitcoin. Lightning Network's \cite{poon_dryja} payment channels rely on HTLC's to enforce the revocation of older commitment transactions (\ctx{}'s). In our attack scenario, Alice and Bob have a payment channel that they have updated over time using many (\ctx{}'s). Both Alice and Bob keep their own copy of their \ctx{}'s, where their copy can be broadcast by them, and will lock their side of the channel balance with an HTLC and the counterparty's side with a regular payment. This means that in the case of a channel closure, the broadcaster has to wait for his payment, but the counterparty can withdraw funds immediately. Without loss of generality, we can assume that in one such update ($u_1$), the entire channel balance was in Bob's favor, and Alice has zero balance in her favor. In a subsequent update ($u_2$), Alice delivers some goods/services to Bob, and after $u_2$, the entire channel balance is in Alice's favor and Bob has zero balance on his side of the channel. As a part of the Lightning Protocol, during $u_2$'s negotiation, Bob gives Alice the preimage ($p_1$) of a hash that lets her punish him if $u_1$ ever makes it to the blockchain. 

The briber (in our case, Bob) broadcasts an outdated \ctx{} $u_1$ (called Revoked Commitment Transaction in Lightning). This has one output which is an HTLC. He then follows it up by broadcasting the bribing transaction: \bribetxn. Note that the \bribetxn{} is timelocked and should be invalid till the timelock expires. The victim (Alice in our case), sees $u_1$ on the blockchain, and using her knowledge of the revocation preimage, sends the corresponding \sellertxn{} (called Breach Remedy Transaction in Lightning) to the pool of transactions to be included in the blockchain. At this time, \sellertxn{} should be valid as it has no timelock on it. But if all miners wait for the \bribetxn{}'s timelock to expire, and during that time ignore the \sellertxn{}, the bribing attack is successful. The amount that goes from the \bribetxn{} to the miner does not matter to Bob because he already has the equivalent goods/services from Alice for that value. Therefore, he is bribing with what he has already spent.

Lightning Network uses HTLC's to also implement payment hops from, say, Alice to Bob through Carol - where Alice and Bob do not have a direct payment channel between each other, but both have a channel to Carol. HTLC's are used here to ensure that Carol can use her channels to send funds from Alice to Bob without Carol's own funds being put at risk. Either the entire payment goes through from Alice to Bob through Carol (who gets the routing fees), or the entire payment is aborted, and all parties retain their own pre-payment balances. Using a series of messages \cite{bolt_3}, Alice, Bob, and Carol communicate using an off-chain protocol and negotiate a series of commitment transactions that each have an additional HTLC that sends the new payment from Alice to Bob through Carol. These HTLC's have a different payment specific secret preimage and its associated hash that locks the hashlock arm of the HTLC. They also have a lower timeout value (compared to the channel's timeout value) that refunds this particular payment back to the source in case any other node along the payment route aborts the payment. These hops do not affect the bribing attack model: an outdated \ctx{} can still be broadcast by the briber and the victim has to respond.

\subsection{Atomic Swaps}
As we saw in the background section in Chapter \ref{chap:background}, Atomic Swaps are a way to exchange cryptocurrencies between two public blockchain systems (say, between Bitcoin and Litecoin, or between Bitcoin and Bitcoin) without involving a trusted third party \cite{herlihy2018atomic}, \cite{atomic_swaps_american_call_options}. Here, we describe TierNolan's classic Atomic Swap construction \cite{atomic_swap} based on the HTLC's used in it. Alice and Bob have their own \htlctxn's in the blockchains whose assets they have. These \htlctxn's will enable corresponding \sellertxn{}'s to the other party and \refundtxn{}'s to themselves. Alice initiates her side of the swap by publishing an HTLC on her blockchain which has a timelock of $2\cdot t$ and hash of a secret preimage that only she knows. Bob accepts the swap by publishing his own HTLC on his blockchain with a timelock of $1\cdot t$ and the same hash whose preimage he \emph{does not} know. Alice then redeems Bob's HTLC by revealing her secret through a \sellertxn{} on Bob's blockchain. Bob's knowledge of this secret (by monitoring Bob's public blockchain) enables Bob to publish his own \sellertxn{} on Alice's blockchain, thereby completing the swap. 

In the atomic swap described above, Alice can try to censor Bob's \sellertxn{} with her own \bribetxn{} on her blockchain that lets her keep assets on Bob's blockchain, and leave most of her bribing profits on her own blockchain to miners. This way, Alice only profits if her attack succeeds, and has no possibility of a loss. Ideally, this should not be possible because Bob's \sellertxn{} is valid from the moment he gets to know of Alice's secret preimage, and Alice's \bribetxn{} is invalid at that time. But if all miners are made aware of Alice's \bribetxn{}, the bribing attack might succeed.

\section{Analysis}
In this section, we analyze the parameters under which this bribing attack is successful. As Alice and Bob both have to agree on the HTLC for it to be valid, they can control these parameters to avoid the attack. The HTLC parameters are:
\begin{itemize}
    \item $T$: denotes the number of blocks needed until the \bribetxn{} becomes valid. This is the HTLC's timelock expressed in terms of number of blocks.
    \item $f$: fee offered by Alice to miners to confirm her \sellertxn{}.
    \item $b$: bribe offered by Bob to miners to confirm his \bribetxn{}. Note that $b$ is not explicitly called out in the transaction because all unclaimed outputs of a transaction go to the miner who confirms it. Typically, $b > f$.
\end{itemize}
There are parameters of the network that Alice and Bob do not control. These are the percentages of the total hashpower that identifiable miners control. Unidentifiable miners are grouped in a catch-all group. Miners are identified based on their coinbase transaction indicators (see section \ref{section:mining_pools} for more details). Let there be $n$ miners $M_j$, $1\leq j \leq n$, each with a fraction $p_j$ of the total hashpower.

\subsection{Assumptions}
\begin{itemize}
    \item Miners are rational and choose the most profitable strategy on what transactions to include in their blocks while conforming to the consensus rules of Bitcoin. Their goal is to maximize expected payoff, and not mine altruistically. 
    \item Miners are also rational in the sense that they will not choose a dominated strategy when they can choose one that is not. A strategy $s$ is dominated by strategy $s'$ if the payoff for playing strategy $s$ is strictly greater than the payoff for playing $s'$, independent of other players' strategies.
    \item Miners do not create forks. If a transaction is included in a valid block, miners build the blockchain on top of that block.
    \item Relative hashpowers of miners is common knowledge. Currently, almost all Bitcoin blocks are mined by mining pools, and almost all of these blocks have an identifiable signature in the coinbase transaction that allows them to identify this relative share of hashpowers. 
    \item Relative hashpowers of miners stay constant over the duration of the bribing attack.
    \item The attacker and the victim of the bribery attack have no hashpower of their own. 
    \item Timelocks are expressed in number of blocks, and we are thus operating in a setting where block generation is equivalent to clock ticks.
    \item Block rewards and fees generated by transactions external to our setting are constant and have no bearing on the attack itself. 
    \item All miners can see timelocked transactions that are valid in the future. Currently, the most popular Bitcoin implementation, Bitcoin Core, does not allow timelocked transactions that are ``valid in the future'' to enter its pool. Consequently, it does not forward such transactions through the peer to peer network. This is not a consensus rule, but rather an efficiency gain whereby allowing only valid transactions to enter the pool and propagate across the peer to peer network reduces network and memory load. We assume that \sellertxn{} and \bribetxn{} are visible to all miners immediately after they are broadcast by their respective parties. Also, some mining pools run ``transaction accelerator'' services where they cooperate with other mining pools to get visibility to transactions that pay an extra fee (on top of the blockchain fee). We assume that malicious buyers have access to such services.    
\end{itemize}

\subsection{Setting}
We analyze this attack by modeling the sequence of blocks being mined as a (Markov) game, called the \emph{bribing game}. A bribing game has $n$ miners, and runs in $T+1$ sequential stages. Stages represent periods between two mined blocks. In each stage, every miner has two possible actions: \follow{} or \refuse{} (corresponding to a miner excluding the \sellertxn{} from the miner's block template or not). After all miners play their action, a single miner is randomly selected as the leader of the stage. In other words, after all the miners have decided on their block template, a single miner wins the proof of work lottery and this miner's block extends the blockchain. 

Let $B_1, B_2, \ldots, B_{T}$ be all the blocks that can include \sellertxn{}. Let $B_{T+1}$ be the block that includes \bribetxn{}. Note that \bribetxn{} cannot be included in $B_1, B_2, \ldots, B_{T}$ as it's not valid then. Let $\mathcal{E}_{i,j}$ denote the event that miner $j$ is selected as the leader of stage~$i$. The events $\mathcal{E}_{i,j}$ are independent of each other and the actions taken by miners. $\mathcal{E}_{i,j}$ represents block $B_i$ being mined by miner $M_j$. In addition, the \emph{selection probability} of miner~$j$ for block~$i$ is given by:  
\[
  \forall i,j \quad Pr(\mathcal{E}_{i,j})= p_j,
\]
which corresponds to the hashpower of miner $M_j$. Each stage is in either of two states: \emph{active} or \emph{inactive}. The game starts in an active stage (i.e., the first stage is active). Stage~$i$, $i>1$, becomes inactive if the leader of stage $i-1$ plays the action \refuse{} (corresponds to including \sellertxn{}), or if stage $i-1$ is already inactive. Therefore, if one stage becomes inactive, all the following stages become inactive. This intuitively makes sense because once \sellertxn{} is confirmed, it stays confirmed in subsequent blocks and more importantly, \bribetxn{} is invalid after that. The payoffs for each stage~$i$ are determined by whether $1\leq i\leq T$ or if $i=T+1$.
\begin{itemize}
    \item $1\leq i\leq T$: If the leader plays \refuse{}, the payoff is $f>0$. If the leader plays \follow{}, the payoff is 0. Non-leaders' payoff is always 0.
    \item $i=T+1$: Leader's payoff is $b>0$. Non-Leaders' payoff is 0.
\end{itemize}

Let us call a miner $M_j$ \strong{} if $p_j\geq \frac{f}{b}$; otherwise we call $M_j$ \weak{}. Note that the bribing attack is successful if all miners follow the bribe  (i.e., they always ignore \sellertxn{}). This corresponds to the strategy profile in which all miners play the action \follow{} in all stages. Without loss of generality, there are two possible distributions of hashpowers among miners:

\begin{itemize}
    \item All miners are strong; i.e., $p_j\geq \frac{f}{b}$ for $1\leq j\leq n$.
    \item At least one miner is weak; i.e, $\exists p_j$ s.t. $p_j < \frac{f}{b}$ for $1\leq j\leq n$.
\end{itemize}

In the next sections, we analyze both of these distributions.

\subsection{All miners are strong} \label{ss:all_miners_strong}

\begin{lemma}
\label{lem:strong}
  If all miners are \strong{} (i.e., $p_j\geq \frac{f}{b}$ for $1\leq j\leq n$), then the strategy profile in which every miner plays \follow{} in all stages is an equilibrium.
\end{lemma}
\begin{proof}
    Consider Miner $j$ ($M_j$), and assume that all other miners follow the bribe in all stages. We show that following the bribe in all stages is the best response for $M_j$ as well. If $M_j$ follows the bribe in all stages, they will earn $p_j\cdot b$ in expectation. This is because, when all miners play \follow{} in all stages, stage $T+1$ will be active, and its leader, which is $M_j$ with probability $p_j$, earns $b$. 
    
    If $M_j$ plays \refuse{} with non-zero probability in at least one stage. Let $x>0$ be the probability that stage $T+1$ becomes inactive as the result of $M_j$'s actions. In other words, $x$ is the probability that $M_j$ plays \refuse{} in a Stage $1\leq i \leq T$ in which they are selected as the leader. Note that other miners cannot make stage $T+1$ inactive as they always play \follow{} and only $M_j$ is including \sellertxn{} in their block template. The expected payoff of $M_j$ is, therefore, $x\cdot f+(1-x)\cdot p_j\cdot b$, which is not more than $p_j\cdot b$, because $p_j \geq \frac{f}{b}$ and $x>0$. 
\end{proof}

Note that when all miners are strong, the equilibrium shown in Lemma~\ref{lem:strong} (which favours bribery) exists no matter how large $T$ is. As of this writing, the average fees for Bitcoin transactions since the beginning of 2019 is around 0.00003 BTC (author's own analysis of the Bitcoin blockchain). The average balance held by a lightning channel is 0.026 BTC \cite{1ml}. If we use these values, we get the equilibrium stated in Lemma~\ref{lem:strong} exists if each miner has over 0.115\% of the total hash power of the entire Bitcoin network. Due to the permissionless and anonymous nature of Bitcoin, however, we can never be sure that the weakest miner has a hash power above 0.115\% of the total hash power. However, we can inspect the Bitcoin blockchain to guesstimate the distribution of hashpowers among known mining pools, and recommend channel parameters based on that. We treat this in more detail in section \ref{section:solutions}. Next, we consider the case where at least one miner is weak. We show that, in this case, the value of $T$ matters.

\subsection{One miner is weak} \label{ss:one_miner_weak} 
Recall that when a stage becomes inactive, all its followup stages become inactive as well. Moreover, all miners receive zero payoff in an inactive stage, irrespective of what they play. Note that, for every miner (weak or strong), playing \follow{} at state~$T+1$ is the strictly dominant strategy if stage~$T+1$ is active. This is because the expected payoff of a miner in an active stage~$T+1$ is $p_jb$ if they play \follow{}, and $p_jf$ (which is smaller than $p_jb$) if they play \refuse{}.
In the next lemma, we show that in active stages other than stage~$T+1$, playing \refuse{} is the strictly dominant strategy for weak miners.

\begin{lemma}
\label{lem:one_miner_weak}
  In any active stage~$i$, $1\leq i \leq T$, playing \refuse{} is the strictly dominant strategy for any weak miner.
\end{lemma}
\begin{proof}
    A miner earns $b$ if stage~$T+1$ is active and this miner is selected as the leader of stage~$T+1$. Therefore, the probability that a Miner~$j$ ($M_j$) earns $b$ is at most $p_j$. From the definition of weakness, for $M_j$, we have $p_j \cdot b<f$. So, if stage~$T+1$ is active, the weak miner gets an expected payoff less than $f$. Additionally, in stages~$<T$, the probability that a miner earns $f$ is strictly less than one, because, no matter how large $T$ is, there is always a non-zero chance that the miner never gets selected as a leader. Therefore, across all stages up to and including stage~$T+1$, the expected payoff of a weak miner is always strictly less than $f$. 
    
    Assume $M_j$ is weak (i.e., $p_j<\frac{f}{b}$), and plays \follow{} in an active stage~$i$, $1\leq i \leq T$. We now show that playing \refuse{} in stage~$i$ will improve her payoff. Suppose $M_j$ plays \refuse{} instead of \follow{} in the active stage~$i$. If $M_j$ is not selected as the leader of stage~$i$, then the game remains the same as the case where $M_j$ played \follow{}. If $M_j$ is selected as the leader, however, they will earn $f$. This is an improvement over the \emph{expected payoff} of $M_j$ from the previous paragraph, which is strictly less than $f$.
 
\end{proof} 

\subsection{The elimination of dominated strategies} \label{ss:iterated_elimination_of_dominated_strategies} 
By Lemma~\ref{lem:one_miner_weak}, playing \refuse{} is the strictly dominant strategy for every weak miner; any other strategy is strictly dominated. Hence, we can simplify the analysis of the bribing game by eliminating strictly dominated strategies. Let us call a bribing game \emph{safe} if after eliminating strictly dominated strategies, the only action left for each miner (strong or weak) in stage one is to play \refuse{}. If every miner plays \refuse{} in stage one, the game is effectively over as other stages become inactive immediately after, with \sellertxn{} confirmed and \bribetxn{} becoming invalid.

Recollect that, if all the miners are strong, the bribing game is not safe no matter how large $T$ is (Lemma~\ref{lem:strong}). By the next theorem, however, the game is safe if there is at least one weak miner, and $T$ is large enough.

%Note that stage~$T+1$ of a $t$-safe bribing game is inactive, $t\geq1$, stage~$T+1$ inactive, thus makes the bribing attack unsuccessful
%
%When $T$ is large enough, this leads us to a unique equilibrium, as proven in the following theorem.
%In this unique equilibrium, every miner plays \refuse{} in the first stage.
%This turns stage~$T+1$ inactive, thus makes the bribing attack unsuccessful. 
%Note that, this is in contrast to the case where all miners are strong, which has an equilibrium in which 
%all miners always follow the bribe.

\begin{theorem}
\label{thm:largeT}
  Suppose there is at least one weak miner, and
  \begin{equation}
  \label{equ:minT}
  T> \frac{\log \frac{f}{b}}{\log (1-p_w)}
  \end{equation}
   where $p_w$ is the sum of the selection probabilities of weak miners. Then, the bribing game is safe. 
\end{theorem}
\begin{proof}
  By Lemma~\ref{lem:one_miner_weak}, playing \refuse{} is the strictly dominant strategy for every weak miner in
  each stage~$i$, $1\leq i \leq T$. By eliminating the dominated strategies of weak miners, we get a smaller game in which weak miners play \refuse{} in every stage~$i$, $1\leq i \leq T$.
  
  Consider a strong miner $M$, who plays \follow{} in stage~1. Their reward for playing \follow{} is only possible at stage~$T+1$. Let $\alpha$ be the probability that stage~$T+1$ will be active. Since weak miners only play \refuse{} in the first $T$ stages, we get 
  \[
  \begin{split}
    \alpha
    &\leq (1-p_w)^T\\
    &\leq (1-p_w)^{\frac{\log \frac{f}{b}}{\log (1-p_w)}}\\
    &\leq \frac{f}{b(1-p_w)}
  \end{split}
  \]
  where $ (1-p_w)^T$ is the probability that no weak miner is selected as a leader in the first $T$ stages. Thus, the expected payoff of $M$ at stage~$T+1$ is less than 
  \[
    \frac{f}{b(1-p_w)}\cdot (1-p_w).b=f
  \]
  where $ \frac{f}{b(1-p_w)}$ is an upper bound on the probability that stage~$T+1$ is active, and $(1-p_w)$ is an upper bound on the probability that $M$ is selected as the leader of stage~$T+1$. Note that the probability that $M$ earns $f$ prior to stage~$T+1$ is strictly less than one. Therefore, at the beginning of stage~1, the expected payoff of $M$ is strictly less than $f$. Now, if $M$ plays \refuse{} (instead of \follow{}) in the first stage, we will have two possibilities. First possibility is that $M$ is selected as the leader of stage~1, in which case $M$ earns $f$, 
  which is strictly more than its expected payoff. In the second possibility where $M$ is not selected as the leader of stage~1, the game remains identical to the original case where $M$ plays \follow{}. This implies that $M$ is better off playing \refuse{} in the first stage, which concludes the proof. We remark that this result does not imply that $M$ is better off playing \refuse{} in every stage. In fact, as the game proceeds to new stages, the expected payoff of $M$ can change, and  $M$ may choose to play \follow{}.
  
\end{proof}

\subsection{The elimination of dominated strategies of strong miners}
A bribing game with parameters $f$ and $b$ may be safe for a significantly smaller $T$ than what is given in Theorem~\ref{equ:minT}. In its proof, we eliminated only strictly dominated strategies of weak miners. In principle, we can continue the process by eliminating strictly dominated strategies of strong miners as well. To do so, we can first sort the strong miners according to their selection probabilities. Starting with the strong miner with the smallest selection probability, and an upper bound of $T$ from Theorem~\ref{equ:minT}, we can calculate the minimum number of initial stages in which the miner is strictly better off playing \refuse{}. We then eliminate the strictly dominated strategies of that miner, and move to the next strong miner. At the end of this iterated elimination process, if all miners play \refuse{} in the first stage, then the game is proven to be safe. As we iterate from time period 0 to time period $T$, the value of $t$ where all miners play refuse for the \textit{last} time shows us that if we had begun the game at this point, the game would have been safe in the first stage itself. This new starting point of the game results in the new ending point being at $T_{new} = T_{old} - t$. In this new setting, the game is safe in the first stage. 

The \texttt{FIND\_T} procedure receives as input a list of mining hashpowers (leader selection probabilities), and the values of parameters $f$ and $b$. As output, it returns the lowest value of $T$ such that all miners refuse the bribe in the first stage of the game. It uses the inner procedure \texttt{CALCULATE\_BRIBERY\_MATRIX} to determine the behavior of more strong miners at each block when less strong miners' strategies get dominated.

\begin{figure}[H]
\centering
\caption{Iterated Removal of Dominated Strategies}\label{algorithm_1}
\label{bribing:algorithm}
\begin{algorithmic}[1]
\Procedure{CALCULATE\_BRIBERY\_MATRIX}{$\mathbb{P}, f, b, T$}
    \State $\mathbb{B} \gets [][]$ \Comment{  Bribery Matrix where B[j][i] represents whether $miner_j$ follows the bribe at $block_i$}
    \For{$j \gets 0$ to $length(\mathbb{P})$}
        \If{$\mathbb{P}[j] < f/b$}
            \State $\mathbb{B}[j] \gets \underbrace{[1, 1, ... 1]}_{T}$
        \Else
            \State $\mathbb{B}[j] \gets \underbrace{[0, 0, ... 0]}_{T}$
            \For{$t_x \gets 1$ to $T$}
                \State $P_h \gets 1$
                \For {$t_y \gets 1$ to $t_x$}
                    \State $sum \gets 0$
                    \For {$k \gets 0$ to $j$}
                        \State $sum \gets sum + \mathbb{B}[k][t_y] \cdot \mathbb{P}[k]$
                    \EndFor
                    \State $P_h \gets P_h * (1 - sum)$
                \EndFor
                \State $expected\_bribe = P_h * \mathbb{P}[j] * b$
                \If{$f > expected\_bribe$}
                    \State $\mathbb{B}[j][t_x] = 1$
                \EndIf
            \EndFor
        \EndIf
    \EndFor
    \State \textbf{return} $\mathtt{B}$
\EndProcedure
\break
\Procedure{FIND\_T}{$\mathbb{P}, f, b$} \Comment{  P is the array of miners' hashpowers}
    \State \textbf{assert}$(\text{at least 1 value in }\mathbb{P} > f/b)$
    \State $\mathbb{P} = sorted(\mathbb{P})$ \Comment{  Ascending}
    \State $T = \ceil{\frac{\log \frac{f}{b}}{\log (1-p_w)}}$ \Comment{  From Theorem~\ref{thm:largeT}}
    \State $\mathbb{B} = \text{CALCULATE\_BRIBERY\_MATRIX}(\mathbb{P}, f, b, T)$
    \For{$i \gets 1$ to $T$}
        \For{$j \gets 0$ to $length(\mathbb{P})$}
            \If{$\mathbb{B}[j][i] == 0$}
                \State \textbf{return} $T - (i - 1)$
            \EndIf
        \EndFor
    \EndFor
    \State \textbf{return} $T$
\EndProcedure
\end{algorithmic}
\end{figure}


\textbf{Example (Table \ref{table:bribery_matrix}):}
Let's take the case of 4 miners with hashpower shares $\mathbb{P} = [0.1, 0.2, 0.3, 0.4]$, $f = 11, b = 100$. Applying Theorem~\ref{thm:largeT}, we get an upper bound of $T$ to be 21. Running the procedure \texttt{CALCULATE\_BRIBERY\_MATRIX} returns the matrix shown in Table \ref{table:bribery_matrix}, with ``1'' standing for \refuse{} and ``0'' standing for \follow{}. Note that this matrix shows the conservative scenario of T=21 blocks (as given by Theorem~\ref{thm:largeT}. The aim of this algorithm is to find a more aggressive (lower) value of T which we get if we eliminate dominated strategies of strong miners. We now go through the actions of each miner.

\begin{table}[ht]
\caption{Bribery Matrix, Worked Example} 
\centering
\begin{tabular}{|c| c c c c|} 
 \hline
 Blocks & 0.1 & 0.2 & 0.3 & 0.4 \\
 \hline
 Block \#1 & 1 &  1 &  1 &  1 \\
 \hline
 Block \#2 & 1 &  1 &  1 &  1 \\
 \hline
 Block \#3 & 1 &  1 &  1 &  1 \\
 \hline
 Block \#4 & 1 &  1 &  1 &  1 \\
 \hline
 Block \#5 & 1 &  1 &  1 &  1 \\
 \hline
 Block \#6 & 1 &  1 &  1 &  1 \\
 \hline
 Block \#7 & 1 &  1 &  1 &  1 \\
 \hline
 Block \#8 & 1 &  1 &  1 &  1 \\
 \hline
 Block \#9 & 1 &  1 &  1 &  1 \\
 \hline
 Block \#10 & 1 &  1 &  1 &  1 \\
 \hline
 Block \#11 & 1 &  1 &  1 &  1 \\
 \hline
 Block \#12 & 1 &  1 &  1 &  1 \\
 \hline
 Block \#13 & 1 &  1 &  1 &  1 \\
 \hline
 Block \#14 & 1 &  1 &  1 &  1 \\
 \hline
  Block \#15 & 1 &  1 &  1 &  1 \\
 \hline
 Block \#16 & 1 &  1 &  0 &  0 \\
 \hline
 Block \#17 & 1 &  0 &  0 &  0 \\
 \hline
 Block \#18 & 1 &  0 &  0 &  0 \\
 \hline
 Block \#19 & 1 &  0 &  0 &  0 \\
 \hline
 Block \#20 & 1 &  0 &  0 &  0 \\
 \hline
 Block \#21 & 1 &  0 &  0 &  0 \\ 
 \hline
 \end{tabular}
\label{table:bribery_matrix}
\end{table}

The miner with hashpower 0.1 ($p_0$) will play \refuse{} at every block because we have $T > \frac{\log \frac{f}{b}}{\log (1-p_w)}$. The miner with hashpower 0.2 ($p_1$) will play \refuse{} as long as the expected bribe (payable at $T+1$) calculated at a particular block is lower than the fees that they would earn if they mine that block. In this case, $(1- p_w)^t \cdot p_1 \cdot b < f$ till $t = 6$ for values of $f = 11, b = 100, p_w = 0.1$. This means that $p_1$ will start playing \follow{} as we get closer to $t = T$ (specifically when we are 5 blocks away from $T$). The miner with hashpower 0.3 ($p_3$) will play \refuse{} along similar lines, by looking at the actions of miners $p_0$ and $p_1$ over the different blocks. One thing to notice is that at block \#16, $p_2$ will act assuming that $p_0$ and $p_1$ will both play \refuse{}. At block \#17, $p_2$ will act assuming that $p_0$ will play \refuse{} and $p_1$ will play \follow{}. This is implemented in the algorithm by using the 0's and 1's in the bribery matrix and using them as factors in line \#13 of the \texttt{CALCULATE\_BRIBERY\_MATRIX} procedure. This way, on line \#13, we only use miners who play \refuse{} at each block to calculate the expected bribe.

$T_{new}$ is lower than $T$, and now, with just one weak miner, and elimination of dominated strategies of all miners, the game is safe for lower values of $T$. This lower value of $T$ makes the usage of HTLC's more practical and convenient. In the real world, we can give a 5-6 block cushion on top of this, and it will still be significantly lower than the upper bound of $T$.

\section{Solutions}
\label{section:solutions}
In the introduction, we pointed out that the two main applications of HTLC's: Lightning Channels and Atomic Swaps, are both vulnerable to this bribing attack. In this section, we first analyze the Bitcoin blockchain to get an estimate of the hashpower share of known mining pools. This lets us find parameters that can harden the HTLC constructions in each of these applications such that they are not vulnerable to the bribing attack. In the case of Atomic Swaps, to use these parameters, we propose a modification to the classic atomic swap protocol.

\subsection{Mining Pools and their Hashpower Shares}\label{section:mining_pools}

We try to find the weakest known miners in the Bitcoin ecosystem by analyzing the miners of the 16000 blocks from Block \#625000. We know the coinbase transaction indicators of larger mining pools. Using these, we can attribute mined blocks to known mining pools. Looking at these blocks, we can estimate each of these mining pools' share of the total hashpower based on how many blocks they have mined. Mining pools and their hashpower shares are shown in Table~\ref{table:mining_powers}. We see that the weakest known pools are under 1\% of the total hashpower, and this leads to our proposed fixes for both Lightning Channels and Atomic Swaps. 

\begin{table}[ht]
\caption{Hashpower of 16000 blocks from block \#625000} 
\centering
\begin{tabular}{|l|l|l|l|}
\hline
        Mining Pool & Hashpower & Mining Pool & Hashpower \\
\hline
 F2Pool     & 15.7937\% & BTCTOP      & 2.6313\% \\
 PoolIn     & 15.5563\% & NovaBlock   & 0.9500\% \\
 BTC.com    & 12.2688\% & SpiderPool  & 0.6125\% \\
 AntPool    & 12.1625\% & Bitcoin.com & 0.1938\% \\
 Huobi      & 6.5875\%  & UkrPool     & 0.0938\% \\
 58COIN     & 6.3000\%  & SigmaPool   & 0.0750\% \\
 ViaBTC     & 5.7875\%  & OkKong      & 0.0688\% \\
 OKEX       & 5.6437\%  & NCKPool     & 0.0625\% \\
 Unknown    & 4.0687\%  & MiningCity  & 0.0500\% \\
 SlushPool  & 3.8188\%  & KanoPool    & 0.0250\% \\
 Lubian.com & 3.6938\%  & MiningDutch & 0.0187\% \\
 Binance    & 3.5375\%  &             &         \\
\hline
\end{tabular}
\label{table:mining_powers}
\end{table}

\subsection{Lightning}
In the Lightning Network specifications (specifically, from Bolt 2 \cite{bolt_2}), we have the following parameters:
\begin{itemize}
    \item \emph{channel\_reserve\_satoshis}: Each side of a channel maintains this reserve so it always has something to lose if it were to try to broadcast an old, revoked commitment transaction. Currently, this is recommended to be 1\% of the total value of the channel. This is the amount that the cheated party can utilize as extra fees without dipping into their own side of the channel. 
    \item \emph{to\_self\_delay}: This is the number of blocks that the counterparty's self outputs must be delayed in case a channel closes unilaterally from the counterparty's side. In one popular Lightning client: c-lightning \cite{c_lightning}, this is set by default to 144 blocks (approximately 1 day). In another popular Lightning client: LND \cite{lnd}, it is scaled in a range from 1 day to 14 days based on the channel value. 
\end{itemize}
We do not find any documented reasons on why these important parameters are set the way they are. Based on the analysis from Sections~\ref{ss:one_miner_weak} and~\ref{ss:iterated_elimination_of_dominated_strategies}, and the distribution of hashpowers, we can formulate what these values ought to be. 
First, we note that \emph{channel\_reserve\_satoshis} on the victim's side of this bribing attack can be used by the victim to increase their fees to thwart the attack. We posit that \emph{channel\_reserve\_satoshis} being at 1\% is reasonable, given that there are many known miners whose hashpower is less than 1\% of the total hashpower of all miners. If it were lower than, say, 0.03\%, as per Section \ref{ss:all_miners_strong}, the channel would be always vulnerable to this bribing attack.     

We then set $\frac{f}{b}$ to be 0.01, and calculate the total \emph{weak} hashpower to be 0.0215 (from Table~\ref{table:mining_powers}). Based on Theorem \ref{thm:largeT}, we get $T > 212$ blocks. This is larger than the suggested default of \emph{to\_self\_delay} at 144 blocks. So, if the channel operator is paranoid, they can set \emph{to\_self\_delay} to this higher value of 212. We can plug in the hashpowers from Table \ref{table:mining_powers} into Algorithm \ref{algorithm_1}, with $f = 1$ and $b = 100$ and we get a value of $T = 54$ blocks. If the channel operator is \emph{\#reckless} and believes that miners eliminate strictly dominated strategies of other miners (a stronger assumption than just assuming that weak miners exist), they can open channels with this much lower timelock value. Note that these values do not actually impact the usage of the Lightning Network, but are merely security parameters that ensure that both parties are adequately protected in case the other party decides to bribe miners. 

\subsection{Atomic Swaps}
Atomic Swaps (as described in \ref{chap:background}) that have Bitcoin on one side need to take Bitcoin's block time of 10 minutes into account. Even if the other blockchain in question (say Litecoin) has faster block generation, till Bitcoin's transactions are not confirmed, the atomic swap in question cannot be considered executed. Commercial platforms like Komodo \cite{komodo} use 15,600 seconds (26 blocks) as the HTLC's timelock value when they setup swaps between Bitcoin-like currencies or ERC-20 style tokens. Other works \cite{atomic_swaps_american_call_options}, \cite{atomic_swaps_bitmex}, \cite{htlcs_considered_harmful} have suggested that a timelock period of 1 day (144 blocks) is a good default. 

Based on Theorem \ref{thm:largeT}, we get $\frac{f}{b} = 0.68$ at $T = 26$ blocks and $\frac{f}{b} = 0.122$ at $T = 144$ blocks. A fee to bribe ratio of 0.68 (for $T = 26$ blocks) is quite high. This suggests that $T = 26$ blocks does not provide enough security for reasonable values of fee to bribe ratios. At 144 blocks, we have a reasonable fee to bribe ratio of 0.122. 

Unlike Lightning channel's \emph{channel\_reserve\_satoshis}, due to its inherently asymmetric nature, there is no simple way to encode this extra fee in the atomic swap itself. Alice has to convince Bob upfront that she will not attempt the bribing attack when it is Bob's turn to redeem his side of the swap. One way of achieving this is for Bob to offer a lower value than what Alice wants. This way, if Alice attempts the bribery attack, Bob can increase his \sellertxn{} fees to the amount dictated by Theorem \ref{thm:largeT} or Algorithm \ref{algorithm_1}. But if Alice does not attempt to bribe, this atomic swap setup is unfair to her as she is getting a lower value from Bob than what she is offering to Bob.

To solve this, we present an extension to the classic Tier Nolan Atomic Swap protocol that allows a way for Alice to include extra fees in the swap for Bob to use to ``counter-bribe'' \textit{only} if Alice attempts to bribe.

\subsection{Risk Free Atomic Swaps}
\label{bribing:risk_free_atomic_swap}
Here, as with the classic Tier Nolan swap, Alice creates a (random) secret preimage and hashes it to get her ``locking string''. Alice creates a transaction ({\texttt{ALICE\_TX1}}) that commits her swap amount such that Bob can claim this amount only if he knows the preimage. The ``refund'' part of this transaction, instead of sending the amount back to Alice after a timelock, sends it to a multisig controlled by both Alice and Bob. Alice also creates a second transaction ({\texttt{ALICE\_TX1}}) that uses this multisig controlled output as its first input, and another unrelated input from Alice which adds the extra fees required to make the swap risk-free. The total output of this second transaction is sent to Bob \textit{only} if he has the secret preimage, or to Alice after a timelock. This pair of transactions is created by Alice; the second transaction is pre-signed by Bob and needs to be held by Alice before she broadcasts the first transaction. 

\begin{figure}[hbt!]
    \centering
    \caption{Risk Free Atomic Swaps}
    \label{fig:risk_free_atomic_swap}
\begin{align*}
    \texttt{ALICE\_TX1} &= [(P|\sigma_a)] \mapsto [(P|(\sigma_a \land \sigma_b) \lor (\sigma_b \land H_{s}))] \\
    \texttt{ALICE\_TX2} &= [(f|\sigma_a), \underbrace{(P|(\sigma_a \land \sigma_b))}_{\texttt{ALICE\_TX1}}] \mapsto \\ 
    &\phantomrel{=} {} [((f+P)| ((\sigma_b \land H_{s}) \lor (\sigma_a \land \Delta))]\\ 
    \texttt{ALICE\_BRIBE\_TX} &= [\underbrace{(f+P)|(\sigma_a \land \Delta)}_{\texttt{ALICE\_TX2}})] \mapsto [(f+P)| \emptyset]\\  
    \texttt{ALICE\_REFUND\_TX} &= [\underbrace{(f+P)|(\sigma_a \land \Delta)}_{\texttt{ALICE\_TX2}})] \mapsto [(f+P)| \sigma_a]\\  
    \texttt{BOB\_SWAP\_TX} &= [\underbrace{P|(\sigma_b \land s)}_{\texttt{ALICE\_TX1}})] \mapsto [P| \sigma_b]\\
    \texttt{BOB\_COUNTER\_BRIBE\_TX} &= [\underbrace{(f+P)|(\sigma_b \land s)}_{\texttt{ALICE\_TX2}})] \mapsto [P| \sigma_b]\\
\end{align*}
\end{figure}

Based on whether Alice or Bob abort the swap, or Alice bribes miners, or Alice and Bob complete a normal swap, a combination of these transactions will be broadcast on the both blockchains by Alice and/or Bob as depicted by the flow chart.

Note that the second blockchain transactions are unchanged from the classic Atomic Swap protocol. This is because, unlike Lightning channels, in an Atomic Swap, only the swap initiator (in this case, Alice) can attempt to cheat by bribing the first blockchain's miners after she claims her side of the swap on the second blockchain. So, the modification to the classic swap that brings in the ``counter bribe fees'' is done only on Alice's side of the swap as shown above with the intermediate multisig.

% Define block styles
\tikzstyle{decision} = [diamond, draw, 
    text width=2.5em, text badly centered]
\tikzstyle{b1wideblock} = [rectangle, draw, fill=red!20, 
    text width=20em, text centered, rounded corners, minimum height=4em]
\tikzstyle{b1midblock} = [rectangle, draw, fill=red!20, 
    text width=12em, text centered, rounded corners, minimum height=4em]
\tikzstyle{b1block} = [rectangle, draw, fill=red!20, 
    text width=7em, text centered, rounded corners, minimum height=4em]
\tikzstyle{b2block} = [rectangle, draw, fill=blue!20, 
    text width=5em, text centered, rounded corners, minimum height=4em]

\tikzstyle{line} = [draw, -latex']

\begin{figure}
\centering
\caption{Risk Free Atomic Swap; red = first blockchain; blue = second blockchain;}
\begin{tikzpicture}[node distance = 2cm, auto]
    % Place nodes
    \node [b1wideblock] (init) {Alice prepares \texttt{ALICE\_TX1} and \texttt{ALICE\_TX2}; Alice gets \texttt{ALICE\_TX2} presigned by Bob};
    \node [b1block, below left of=init, node distance=2.5cm] (1) {Alice broadcasts \texttt{ALICE\_TX1}};
    \node [decision, below of=1, node distance=2.25cm] (2) {Bob aborts?};
    \node [b1midblock, right of=2, node distance=4cm] (3) {Alice broadcasts \texttt{ALICE\_TX2} and \texttt{ALICE\_REFUND\_TX}};
    \node [b2block, below of=2, node distance=2.25cm] (4) {Bob broadcasts ``init''};
    \node [decision, below of=4, node distance=2.25cm] (5) {Alice aborts?};
    \node [b2block, right of=5, node distance=4cm] (6) {Bob broadcasts ``refund''};
    \node [b2block, below of=5, node distance=2.25cm] (7) {Alice broadcasts ``reveal''};
    \node [decision, below of=7, node distance=2.25cm] (8) {Alice bribes?};
    \node [b1midblock, right of=8, node distance=4cm] (9) {Alice broadcasts \texttt{ALICE\_TX2} and \texttt{ALICE\_BRIBE\_TX}};
    \node [b1midblock, below of=9, node distance=2.25cm] (11) {Bob broadcasts \texttt{BOB\_COUNTER\_BRIBE\_TX}};
    \node [b1block, below of=8, node distance=2.25cm] (10) {Bob broadcasts \texttt{BOB\_SWAP\_TX}};
    % Draw edges
    \path [line] (init) -- (1);
    \path [line] (1) -- (2);
    \path [line] (2) -- node [midway] (TextNode) {Yes} (3);
    \path [line] (2) -- node [midway] (TextNode) {No}(4);
    \path [line] (4) -- (5);
    \path [line] (5) -- node [midway] (TextNode) {Yes} (6);
    \path [line] (6) -- (3);
    \path [line] (5) -- node (TextNode) {No}(7);
    \path [line] (7) -- (8);
    \path [line] (8) -- node [midway] (TextNode) {Yes} (9);
    \path [line] (8) -- node [midway] (TextNode) {No}(10);
    \path [line] (9) -- (11);
\end{tikzpicture}
\label{fig:atomic_swap}    
\end{figure}


\section{Related Work}
There are two major strands of censorship attacks in blockchains. Ignore attacks (that incentivize miners to ignore certain transactions) and fork attacks (that incentivize miners to orphan blocks with certain transactions by forking the blockchain).

\subsection{Ignore Attacks}
Ignore Attacks are presented in \cite{temporary_censorship_attack_ethereum}, \cite{pay_to_win}, and \cite{mad_htlc}. In \cite{temporary_censorship_attack_ethereum}, smart contracts in a ``funding blockchain'' are used to censor transactions in a ``target blockchain''. Funding blockchains need to support powerful smart contract primitives to be able to program these attacks -- typically Ethereum is used. Two such attack smart contracts presented in \cite{temporary_censorship_attack_ethereum} are Pay-per-Miner and Pay-per-Block. In Pay-per-Miner, every miner gets a bribe at the end of the bribing period if the bribing attack succeeds, even if the miner followed the bribe or not. A weak miner could refuse the bribe, and attempt to mine with the \sellertxn{}, but not succeed in mining a block. This miner would still be eligible for the bribe at the end. This contract does not consider a weak miner's lower probability of mining the final block with the bribe and hence, overpays. In Pay-Per-Block, every miner is paid incrementally per block during the bribing period. This attack also bribes weak miners who go against the bribe, and thus have a higher expected reward at the end of the bribing period. Both these attacks would get better if miners could cryptographically prove to the smart contract that they are following the bribe.

Concurrent to our work, a similar timelocked bribing attack is presented in \cite{mad_htlc}. They consider the situation where all miners are strong (i.e., $p_j\geq \frac{f}{b}$ for all miners $1\leq j\leq n$), and like us, they conclude that the bribing attack will be successful and is independent of the bribing period $T$. To alleviate this situation where all miners are strong and bribing attacks could happen, they propose a modified construction of the HTLC called MAD-HTLC (Mutually Assured Destruction HTLC). MAD-HTLC adds a second transaction chained to the HTLC with a collateral from the bribing counterparty to ensure that they have something to lose if they attempt to bribe. However, \cite{mad_htlc} does not consider weak miners, or elimination of dominated strategies - which we show lead to HTLC parameters that can be adjusted to safeguard against this bribing attack with any distribution of miner hashpowers and values of $f$ and $b$. Our approach also doesn't need a modification to the HTLC construction and the associated collateral and extra transaction costs.

Transaction Pinning \cite{transaction_pinning} tries to make a transaction inherently unprofitable to mine, independent of any future bribe. The attacker, who can validly spend one of the target transaction's outputs broadcasts multiple low fee-rate transactions that spends their path of the target transaction. This makes the entire transaction package unprofitable to mine, thereby censoring the first transaction, which the victim can spend through another path. To remedy this, the victim can use CPFP carve-outs \cite{cpfp_carveout} to bump up the fee-rate of the censored transaction and still get it confirmed by a miner. To enable this, Lightning Channels will allow ``anchor outputs'' \cite{anchor_outputs} to let either party bump up their fees without being blocked by the counterparty.

These types of Ignore Attacks rely on being able to setup and communicate incentives (in the present, or in the future) to miners such that the most profitable strategy for each miner is to wait for the incentive. Whether these incentives succeed or not, depends on the current value available to miners, the future value promised to miners, and the ability of miners to be able to extract these values. Unlike previous research, our work takes into account \textit{all} these parameters.

\subsection{Fork Attacks}
Fork Attacks go back a long way, with the earliest one discussed on bitcointalk.org being \emph{feather forking} \cite{feather_forking}. In this attack, a miner wants to censor a specific transaction and announces on some public bulletin board that they will not add blocks on top of any block that contains this specific transaction. If this miner has a reasonable chance of getting a block, other rational miners will follow them instead of mining ``normally'' and hence forgo the fees of the censored transaction. Feather forking is also analyzed under the Pay-per-Commit contract in \cite{temporary_censorship_attack_ethereum}. Feather forking relies on a miner committing to the attack, and this being common knowledge among all miners. This attack relies on both a funding cryptocurrency blockchain to set up the attack and a way to communicate with all miners that the attack is going to happen.  

Miners can be also incentivized to fork the Bitcoin blockchain with ``Whale Transactions'' \cite{whale_transactions}. Here, the attacker waits for a target transaction to be confirmed to a sufficient depth to get the corresponding goods and services from their victim. After that, the attacker tries to fork the blockchain by successively broadcasting transactions that have high fees (whale transactions) and also reverse the target transaction. These whale transactions are then included in blocks of the blockchain fork that rational miners might follow. The authors evaluate the relationship between confirmation depth, the attacker's secret mining lead, the attacker's hashpower, the whale transaction fees and whether these attacks are profitable. External smart contracts on platforms like Ethereum can be used \cite{smart_contracts_for_bribing}, \cite{pay_to_win} to incentivize Bitcoin miners to abandon the honest blockchain suffix and mine on top of a briber's fork. In \cite{pay_to_win}, the attacker chooses the set of transactions to be mined for each block, and hands it out to miners through the smart contract. This is similar to how mining pools operate. Miners get rewarded in the ``funding cryptocurrency'' (Ether, in this case). Incentivizing every Bitcoin miner with Ether given the relative size of the two systems seems far fetched to us.

Fork Attacks rely on attackers being able to incentivize rational miners to orphan a reasonable length suffix of the blockchain. The attack succeeds if it is conducted after the primary transaction has been thought confirmed by the victim. Given that most proof-of-work cryptocurrencies have a probabilistic notion of finality, these attacks are feasible. On the other hand, Bitcoin has seen fewer and fewer orphan blocks over time \cite{orphans}, and the possibility of this kind of attack is considerably lower now than they were in, say, 2015. 

\section{Conclusion}
In this chapter, we observe that HTLC's are vulnerable to an ``in-band'' bribing attack where the HTLC initiator (buyer, in our case) can receive goods and services offline and then prevent the seller from getting their due share by bribing miners. This bribe can only work if the ``time value'' of waiting for the bribe is worthwhile for all miners. A rather self-evident observation is that when the timelock on the bribe expires and the bribe transaction is still valid, it will be claimed in the immediate next block as the fee on it is considerably higher than normal transaction fees. Additionally, stronger miners are likely to mine any specific block - and therefore more likely to mine the block in which the bribe is valid and available. Therefore, we posit that weaker miners will ignore the bribe altogether and will attempt to mine the seller's transaction while the timelock holds and the fee on the seller's transaction is good enough. This leads us to the relationship between the fee to bribe ratio and the distribution of miners' hashpowers. Based on this analysis, we propose Lightning Channel parameters that make them resistant to this kind of bribing attack. In Atomic Swaps, our analysis also proposes a fee for the victim to safeguard themselves. To enable that, we propose a modification to the classic Tier Nolan Atomic Swap protocol that can bring in this fee into the swap and still keep it fair for both parties.